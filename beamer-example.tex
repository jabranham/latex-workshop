\documentclass{beamer}

%\usetheme{split}
\usetheme{Malmoe}




%fonts:
%\usepackage[math]{anttor}
\usepackage[T1]{fontenc}
\usepackage{kpfonts}
\usepackage{float,subfig}
\usepackage{wrapfig}
\usepackage{tikz,pgfplots}






%\usepackage{beamerthemesplit} // Activate for custom appearance

\title{A Presentation on Presentations Using Beamer}
\author{ J. Alexander Branham \& Henry Pascoe}
\institute{The University of Texas at Austin} 
\date{\today}

\begin{document}

\frame{\titlepage}




\section{Introduction}

\begin{frame}
\frametitle{What is Beamer?}

\begin{itemize}
\item Beamer is similar to Microsoft PowerPoint 
\pause
\item It makes nice looking slides
\end{itemize}
\end{frame}

\begin{frame}
\frametitle

you can also add math to your slides with ease:

\[y=mx+b\]

\end{frame}

\begin{frame}
\frametitle{Errors}

Sometimes (often even) LaTeX will give you errors

\begin{itemize}

\item When this happens there are a few things to check 
\begin{itemize}
\item did you make sure to put dollar signs around symbols?

\item did you end all of your environments?

\item look at the error message, often it will give you a line number where the error is. 

\item commenting things out until it will compile is a nice way to find errors. 
\end{itemize}
\end{itemize}

\end{frame}


\section{Figures}
\frame{\frametitle{You can also make figures}
\begin{figure}[t]\centering
\begin{tikzpicture}[scale=1.2][font=\tiny]
	\node at (.5,5.15) {1};
	\node at (-.3,4.6) {$a$};
	\node at (1.3,4.6) {$\neg a$};
\draw[,-] (.5,5)--(-1,4);
\draw[,-] (.5,5)--(2,4);
	\node at (-1,3.85) {2};
	\node at (2,3.85) {2};

\draw[] (-1,3.7)--(-1.4,3);
\draw[] (-1,3.7)--(-.6,3);
\draw[very thick] (-1.1,3.5) to [out=270, in=-90] (-.9,3.5);
	\node at (-1,3.2) {$s$};
	\node at (-1.4,2.9) {0};
	\node at (-.6,2.9) {$r_2$};

\draw[] (2,3.7)--(1.6,3);
\draw[] (2,3.7)--(2.4,3);
\draw[very thick] (1.9,3.5) to [out=270, in=-90] (2.1,3.5);
	\node at (2,3.2) {$s$};
	\node at (1.6,2.9) {0};
	\node at (2.4,2.9) {$r_2$};

\node at (2, 2.7) {1};
\node at (-1, 2.7) {1};

\draw[] (-1,2.5)--(-1.4,1.8);
\draw[] (-1,2.5)--(-.6,1.8);
\draw[very thick] (-1.1,2.3) to [out=270, in=-90] (-.9,2.3);
	\node at (-1,2) {$e$};
	\node at (-1.4,1.7) {0};
	\node at (-.6,1.7) {$r_1 + s$};

\draw[] (2,2.5)--(1.6,1.8);
\draw[] (2,2.5)--(2.4,1.8);
\draw[very thick] (2.1,2.3) to [out=270, in=-90] (1.9,2.3);
	\node at (2,2) {$e$};
	\node at (1.6,1.7) {0};
	\node at (2.4,1.7) {$r_1 + s$};

\draw[, -] (-1,1.2)--(-2, .3);
\draw[, -] (-1, 1.2)--(0, .3);
	\node at (-1, 1.4) {T};
	\node at (-1.7, .8) {c};
	\node at (-.3, .8) {$\neg c$};

\draw[, -] (2,1.2)--(1, .3);
\draw[, -] (2, 1.2)--(3, .3);
	\node at (2, 1.4) {T};
	\node at (1.3, .8) {c};
	\node at (2.7, .8) {$\neg c$};
%(a,c)
\node at (-2.2, .1) {$r_1 + s -e$,};
\node at (-2.2, -.1) {$r_2-s$,};
\node at (-2.2, -.3) {$l$};

%(a, not c)
\node at (-.3, .1) {$r_1+s-e-\beta_1-a$,};
\node at (-.3, -.1) {$r_2-s-\beta_2$,};
\node at (-.3, -.3) {$(1-q)(b)$};

%(not a, c)
\node at (1.1, .1){$r_1+s-e$,};
\node at (1, -.1){$r_2-s$,};
\node at (1, -.3){$l$};

%(not a, not c)
\node at (3, .1){$r_1+s-e-\beta_1$,};
\node at (3, -.1){$r_2-s-\beta_2$,};
\node at (3, -.3){$(1-q)(b)$};

\end{tikzpicture}
\end{figure}

}%end frame





\section{Tables}

\begin{frame}
    \frametitle{Tables}
   You can also make tables:
\begin{table}[ht]
\caption{Nonlinear Model Results}
% title of Table
\centering
% used for centering table
\begin{tabular}{c c c c}
% centered columns (4 columns)
\hline
\hline %inserts double horizontal lines
Case & Method\#1 & Method\#2 & Method\#3\\[0.5ex]
% inserts table
%heading
\hline
% inserts single horizontal line
1 & 50& 837 & 970\\
% inserting body of the table
2 & 47 & 877 & 230\\
3 & 31 & 25 & 415\\
4 & 35 & 144 & 2356\\
5 & 45 & 300 & 556\\[1ex]
% [1ex] adds vertical space
\hline
%inserts single line
\end{tabular}


\end{table}

\end{frame}









  \end{document}      
      
        
        
        
        
        


%%% Local Variables:
%%% mode: latex
%%% TeX-master: t
%%% End:
